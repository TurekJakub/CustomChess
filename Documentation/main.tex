\documentclass[12pt]{article}
\usepackage[utf8]{inputenc}
\usepackage{graphicx}
\usepackage{biblatex}
\usepackage{titlesec}
\usepackage{geometry}
\usepackage{setspace}
\usepackage{minted}
\usepackage{biblatex}
\usepackage{listings}

\geometry{
 a4paper,
  left=35mm,
 top=25mm,
 right=25mm,
 bottom=25mm,
 }
\setstretch{1.5}
\addbibresource{biblio.bib}
\bibliography{reference}
\begin{document}
\graphicspath{{./pictures/}}
\titlelabel{\thetitle.\quad}
\renewcommand*\contentsname{Obsah}
\renewcommand{\bibfont}{\small}
\renewcommand{\thesection}{\Roman{section}} 
\renewcommand{\thesubsection}{\thesection.\Roman{subsection}}
\renewcommand{\thesubsubsection}{\thesubsection.\Roman{subsubsection}}
\renewcommand\listoflistingscaption{Seznam příloh}
\renewcommand{\listingscaption}{Příloha}
\begin{titlepage}
\begin{center}
    \vspace{2,5cm}
    \textbf{\Large Gymnázium, Praha 6, Arabská 14}\par
    \Large Obor Programování\par
    \vspace{2cm}
    \textbf{\Huge Ročníková práce}\par
    \vspace{2cm}
    \textbf{\Huge Custom Chess}\par
    \vspace{2cm}
    \includegraphics[]{logo}\par
    \vfill
   \Large Petr Dobiáš, Josef Liška, Jakub Turek \hfill Duben 2023
   
    
     
\end{center}
\end{titlepage}
\newpage{}
\thispagestyle{empty}
\mbox{}
\vfill
Prohlašuji, že jsem jediným autorem tohoto projektu, všechny citace jsou řádně označené a všechna použitá literatura a další zdroje jsou v práci uvedené. Tímto dle zákona 121/2000 Sb. (tzv. Autorský zákon) ve znění pozdějších předpisů uděluji bezúplatně škole Gymnázium, Praha 6, Arabská 14
oprávnění k výkonu práva na rozmnožování díla (§ 13) a práva na sdělování díla veřejnosti (§ 18) na dobu časově neomezenou a bez omezení územního rozsahu.
\newline
V Praze dne \hfill Petr Dobiáš, Josef Liška, Jakub Turek
\newpage{}
\thispagestyle{empty}
\section*{Anotace}
Následující práce pojednává o vývoji aplikace založené na atchitektůře klient-server s využitím programovacích jazyku Java, Python a JavaScript, frameworku Django a databáze MongoDB, která umožnuje uživatelům vytvařet a následně hrát derivace hry šachy, v podobě úpravy figur, hrací desky a podmínek vítězství, online. Tyto derivace mohou uživatelé definovat ručně pomocí textovích soubru, či grafického návrháře v aplikaci. Aplikace se sestává ze serverové části a klientské webové aplikace.
\section*{Abstract}
Budeme to překládat? :D
\newpage
\setcounter{page}{1}
\tableofcontents 
\newpage
\section*{Úvod}
\addcontentsline{toc}{section}{Úvod}
\newpage
\section*{Lorem ipsum}
Lorem ipsum dolor sit amet, consectetur adipiscing elit. Donec auctor, nisl eget aliquam aliquam, nunc nisl aliquet nisl, eget aliquet
\newpage
\printbibliography[heading=bibintoc,title={Reference}]
\newpage
\addcontentsline{toc}{section}{Seznam příloh}
\listoflistings


\end{document}
